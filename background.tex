
\documentclass{article}

\setlength{\oddsidemargin}{0.25in}

\setlength{\textwidth}{6in}

\setlength{\topmargin}{-0.25in}

\setlength{\textheight}{8in}

\begin{document}

\section{CHAPTER ONE}       

\subsection{Background}  
  \paragraph{The Method of Least Squares is a procedure, requiring just some calculus and linear algebra, to determine what the “best fit” line is to the data. Of course, we need to quantify what
we mean by “best fit”, which will require a brief review of some probability and statistics.
A careful analysis of the proof will show that the method is capable of great generalizations. Instead of finding the best fit line, we could find the best fit given by any finite linear
combinations of specified functions. The least squares problem will also find the value of x that makes Ax as close to b as possible. That is Ax ˜b
}
\paragraph{We add squares of b-Ax which is least or minimum and therefore we call it the solution to least square problem.
A least squares solution to Ax˜b is a true solution to ATAx = ATb and every true solution to ATAx = ATb is a least square solution to Ax˜b.

The QR factorization is the most common, and best known solution to least squares problem.
}

\end{document}
