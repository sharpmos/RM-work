\documentclass[12pt, letterpaper]{article}
\begin{document}
\bibliography{good}
\bibliographystyle{ieeetr}
\noindent
    \title{Investigation of the solution of least squares problems using the QR factorization}
    \author{MUSIIMEMARIA EUGENIA 15/U/8383/EVE \\}
    \date{16/04/2017}
    \maketitle
\section{Introduction}
    Least squares problems. \par
    QR factorization of a matrix is the decomposition of a matrix D into a product D=QR of an orthogonal matrix Q and an upper triangular matrix R.\par
    Orthogonal basis is the relation of two lines at right angles to one another and the generalization of this relation into n dimensions.\par
    Orthonormal basis is a square matrix with real entries whose columns and rows are orthogonal unit vectors.\par
\section{Literature review}
    A QR factorization splits the A matrix up into an upper triangular R matrix, and an orthogonal Q matrix. Since only R is really needed to compute the solution, there is no need to actually form and save Q. This reduces the storage required to just that needed to store R and some space for housekeeping.\par
    For the QR factorization, the R matrix starts out in the form of a spanning tree followed by the redundant shots. Most surveys are already in that form or can be put into it with trivial effort. For a collection of shots with no order, you would have to put it into that form, but that can be done in linear time.\par
    After the QR factorization, the upper triangular part of R is now the same as would have computed for the Cholesky factorization. This allows us to use the computed R to get the solution exactly the same way that it would be used for a solution by normal equations.\par
    Consider solution of the linear system Ax = b with A ∈ C m×m nonsingular.\par
    Since Ax = b ⇐⇒ QRx = b ⇐⇒ Rx = Q ∗ b, where the last equation holds since Q is unitary, we can proceed as follows:\par
     1. Compute A = QR (which is the same as A = QˆRˆ in this case).\par
     2. Compute y = Q∗b.\par
     3. Solve the upper triangular Rx = y\par
    •	The QR factorization can also be applied to rectangular systems and it is the basis of Mat lab’s backslash matrix division operator.\par
\title{Investigation of the solution of least squares problems using the QR factorization}
\author{KIKOMEKO MUSA 15/U/6675/PS \\}
\date{16/04/2017}
\maketitle
\section{METHODOLOGY}
Given data $((x1; y1).......(xN; yN))$, we may define the error associated to saying y = ax + b
\paragraph{This is just N times the variance of the data set and It makes no difference whether or not we study the variance or N times the variance as our error, and note that the error is a function of two variables.}

\paragraph{The goal is to find values of a and b that minimize the error
	$a^2y-(ax+b) =(1/N)N∑(n=1)(yn ¡ (axn + b))^2.$
}
\title{\bfseries QR factorization}
\author{BAMUZIBIRE SOLOMON MUKISA,14/U/3938/PS,214015782}
\section{Introduction}
QR factorization of a matrix is the decomposition of a matrix D into a product D=QR of an orthogonal matrix Q and an upper triangular matrix R.
Orthogonal basis is the relation of two lines at right angles to one another and the generalization of this relation into n dimensions.
Orthonormal basis is a square matrix with real entries whose columns and rows are orthogonal unit vectors
\section{problem statement}
The problem this investigation seeks to solve is least squares problems of approximately solving an over determined system of linear equations, where the best approximation is defined as that which minimizes the sum of squared differences between the data values and their corresponding modeled values.

\section{literature Review}
A QR factorization splits the A matrix up into an upper triangular R matrix, and an orthogonal Q matrix. Since only R is really needed to compute the solution, there is no need to actually form and save Q. This reduces the storage required to just that needed to store R and some space for housekeeping.
For the QR factorization, the R matrix starts out in the form of a spanning tree followed by the redundant shots. Most surveys are already in that form or can be put into it with trivial effort. For a collection of shots with no order, you would have to put it into that form, but that can be done in linear time.
After the QR factorization, the upper triangular part of R is now the same as would have computed for the Cholesky factorization. This allows us to use the computed R to get the solution exactly the same way that it would be used for a solution by normal equations.
Consider solution of the linear system Ax = b with A ? C m×m nonsingular.
Since Ax = b ?? QRx = b ?? Rx = Q * b, where the last equation holds since Q is unitary, we can proceed as follows:
1. Compute A = QR (which is the same as A = QˆRˆ in this case).
2. Compute y = Q*b.
3. Solve the upper triangular Rx = y.
The QR factorization can also be applied to rectangular systems and it is the basis of Mat lab’s backslash matrix division operator.


\end{document}