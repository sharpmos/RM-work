\documentclass[10]{article}
\begin{document}
\begin{title}
\huge{\bfseries COMPUTER NETWORKS}
\end{title}
\author{BAMUZIBIRE SOLOMON MUKISA,14/U/3938/PS,214015782}
\section{Introduction}
A computer network is a group of two or more computing devices connected via a form of communication technology. For example Makerere University uses a computer network connected via cables (Ethernet or fiber optical) in order to gain access from file servers and the internet server.
\subsection{Background}
The history of networks is as described below;
To begin with the electrical telegraph and telephone systems that became the foundation for what would become the first computer networks. Telegraphy was driven by the need to reduce sending costs, either in hand-work per message or by increasing the sending rate. While many experimental systems employing moving pointers and various electrical encodings proved too complicated and unreliable, a successful advance in the sending rate was achieved through the development of telegraphs. Then telephony came in and this could transmit analog data, on March 10, 1876, Alexander Graham Bell spoke into his device and said to his assistant, “Mr. Watson, come here, I want to see you.” In doing so, Bell launched the telephone era with the first bi-directional electronic transmission of the spoken word. At least that is how the story typically goes. In 1963, AT&T introduced Touch-Tone, which allowed phones to use a keypad to dial numbers and make phone calls. Each key would transmit a certain frequency, signaling to the telephone operator which number you wanted to call. While much better than the rotary dial, these dial tones were subject to spoofing by what were called “blue boxes.” Using a blue box, you could make free long-distance phone calls. Then the First Computer Networks came into place when military research groups began to investigate the large-scale coordination of digital information in the 1950’s. The Semi-Automatic Ground Environment (SAGE) was a US Air Force project designed to bring together various military data, such as radar data. One developer of SAGE brought the networking ideas to business with the airline reservation system.

\subsection{Objectives}
The report is intended to avail individuals with a brief knowledge of computer networks, where they came from, future trends and their importance.

\subsection{Scope}
This document examines only the trends from telegraphs and telephony that contributed to today is computer networks.

\section{Literature Review}
A cable or DSL Internet connection and a modem will be required to set up the computer network. Connect a DSL modem to a live phone line. DSL modems will connect through a telephone jack using a standard telephone cable, usually included with the modem at the time of purchase. An account with a local service provider will be required.
 A wireless router also will be required to setup a wireless connection, connect the wireless router to the modem. Using the network cable (typically Ethernet), included with the purchase of the wireless router, plug 1 end into the modem and the other end into the first empty port, going from left to right, on the back of the wireless router. The first port is typically assigned a different color than the other Ethernet ports on the router. Plug the other end of the Ethernet cable into the Ethernet port on the DSL or cable modem. Connect the host computer to the wireless router. Using a network USB or Ethernet cable, plug 1 end into the network adapter on the computer and the other end into the next open port on the wireless router. Plug in the power source for the modem, then plug in the power source for the wireless router. Wait a few moments for the devices to boot up.


\section{Methodology}
In this chapter we discuss the methods used to collect data concerning computer and their origin. Existing literature from the internet and text books in relation to the origin of networks will be reviewed and studied to only get information for things that contributed to today’ networks.
Articles, journals and manuals on how to build secure networks will also be reviewed as to give a clear, a professional approach on how to build networks.


\end{document}
